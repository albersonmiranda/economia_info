% Options for packages loaded elsewhere
\PassOptionsToPackage{unicode}{hyperref}
\PassOptionsToPackage{hyphens}{url}
%
\documentclass[
]{article}
\usepackage{amsmath,amssymb}
\usepackage{lmodern}
\usepackage{iftex}
\ifPDFTeX
  \usepackage[T1]{fontenc}
  \usepackage[utf8]{inputenc}
  \usepackage{textcomp} % provide euro and other symbols
\else % if luatex or xetex
  \usepackage{unicode-math}
  \defaultfontfeatures{Scale=MatchLowercase}
  \defaultfontfeatures[\rmfamily]{Ligatures=TeX,Scale=1}
\fi
% Use upquote if available, for straight quotes in verbatim environments
\IfFileExists{upquote.sty}{\usepackage{upquote}}{}
\IfFileExists{microtype.sty}{% use microtype if available
  \usepackage[]{microtype}
  \UseMicrotypeSet[protrusion]{basicmath} % disable protrusion for tt fonts
}{}
\makeatletter
\@ifundefined{KOMAClassName}{% if non-KOMA class
  \IfFileExists{parskip.sty}{%
    \usepackage{parskip}
  }{% else
    \setlength{\parindent}{0pt}
    \setlength{\parskip}{6pt plus 2pt minus 1pt}}
}{% if KOMA class
  \KOMAoptions{parskip=half}}
\makeatother
\usepackage{xcolor}
\IfFileExists{xurl.sty}{\usepackage{xurl}}{} % add URL line breaks if available
\IfFileExists{bookmark.sty}{\usepackage{bookmark}}{\usepackage{hyperref}}
\hypersetup{
  pdftitle={Natureza das Transformações no Setor de Produção de Bens Culturais},
  pdfauthor={Alberson da Silva Miranda},
  hidelinks,
  pdfcreator={LaTeX via pandoc}}
\urlstyle{same} % disable monospaced font for URLs
\usepackage[margin=1in]{geometry}
\usepackage{graphicx}
\makeatletter
\def\maxwidth{\ifdim\Gin@nat@width>\linewidth\linewidth\else\Gin@nat@width\fi}
\def\maxheight{\ifdim\Gin@nat@height>\textheight\textheight\else\Gin@nat@height\fi}
\makeatother
% Scale images if necessary, so that they will not overflow the page
% margins by default, and it is still possible to overwrite the defaults
% using explicit options in \includegraphics[width, height, ...]{}
\setkeys{Gin}{width=\maxwidth,height=\maxheight,keepaspectratio}
% Set default figure placement to htbp
\makeatletter
\def\fps@figure{htbp}
\makeatother
\setlength{\emergencystretch}{3em} % prevent overfull lines
\providecommand{\tightlist}{%
  \setlength{\itemsep}{0pt}\setlength{\parskip}{0pt}}
\setcounter{secnumdepth}{5}
\newlength{\cslhangindent}
\setlength{\cslhangindent}{1.5em}
\newlength{\csllabelwidth}
\setlength{\csllabelwidth}{3em}
\newlength{\cslentryspacingunit} % times entry-spacing
\setlength{\cslentryspacingunit}{\parskip}
\newenvironment{CSLReferences}[2] % #1 hanging-ident, #2 entry spacing
 {% don't indent paragraphs
  \setlength{\parindent}{0pt}
  % turn on hanging indent if param 1 is 1
  \ifodd #1
  \let\oldpar\par
  \def\par{\hangindent=\cslhangindent\oldpar}
  \fi
  % set entry spacing
  \setlength{\parskip}{#2\cslentryspacingunit}
 }%
 {}
\usepackage{calc}
\newcommand{\CSLBlock}[1]{#1\hfill\break}
\newcommand{\CSLLeftMargin}[1]{\parbox[t]{\csllabelwidth}{#1}}
\newcommand{\CSLRightInline}[1]{\parbox[t]{\linewidth - \csllabelwidth}{#1}\break}
\newcommand{\CSLIndent}[1]{\hspace{\cslhangindent}#1}
\usepackage{abntex2}
\ifLuaTeX
  \usepackage{selnolig}  % disable illegal ligatures
\fi

\title{Natureza das Transformações no Setor de Produção de Bens
Culturais}
\author{Alberson da Silva Miranda}
\date{2022-05-20}

\begin{document}
\maketitle

\hypertarget{introduuxe7uxe3o}{%
\section*{INTRODUÇÃO}\label{introduuxe7uxe3o}}
\addcontentsline{toc}{section}{INTRODUÇÃO}

A economia está inserida na esfera social e, portanto, determinada por
fenômenos sociais. Suas regras, normas e relações estão, por essa razão,
sujeitas à geografia, tempo e estruturas de
poder\footnote{Aqui me refiro às instituições que ora determinaram o \textit{ethos} vigente, como a igreja, aristrocacia ou o capital, por exemplo.}.
Isso implica que seu funcionamento em raras ocasiões --- ou em nenhuma
--- poderá ser explicado por \emph{leis}, que, por definição, são
imutáveis. E no caso dos bens culturais, há evidências de transformações
profundas nos últimos anos.

Neste estudo, {[}\ldots{]}

\hypertarget{fatos-estilizados-da-induxfastria-da-muxfasica}{%
\section{FATOS ESTILIZADOS DA INDÚSTRIA DA
MÚSICA}\label{fatos-estilizados-da-induxfastria-da-muxfasica}}

\hypertarget{caracterizauxe7uxe3o-teuxf3rica-do-setor-de-produuxe7uxe3o-de-bens-culturais}{%
\section{CARACTERIZAÇÃO TEÓRICA DO SETOR DE PRODUÇÃO DE BENS
CULTURAIS}\label{caracterizauxe7uxe3o-teuxf3rica-do-setor-de-produuxe7uxe3o-de-bens-culturais}}

Para caracterizar o setor de produção de bens culturais, nos muniremos
de BOURDIEU (2007) e HERSCOVICI (1995). Como referência, confrontaremos
o corpo teórico desenvolvido pelos autores com a pressupostos comuns ao
\emph{mainstream} para evidenciar as divergências. Primeiramente,
analisaremos as hipóteses adotadas por HERSCOVICI (1995) acerca do
produto cultural.

\hypertarget{hipuxf3tese-1}{%
\subsection{HIPÓTESE \#1}\label{hipuxf3tese-1}}

\begin{quote}
Suponhamos que não seja possível raciocinar em termos de valor
intrínseco da obra de arte. Isto significa simplesmente que as
apreciações feitas a respeito da obra dependem, simultaneamente, da
época e do grupo social considerado, assim como dos modos de validação
em vigor. A obra só pode ser compreendida e apreciada se for recolocada
no seu contexto histórico e sociológico; a universalidade da obra de
arte é, portanto, limitada por estes fatores. (HERSCOVICI, 1995, p. 30)
\end{quote}

\begin{itemize}
\tightlist
\item
  escrever hipóteses (livro Alain, p.30) (hipotese 1: falar da função
  U(c), h2: Bourdieu, h3: homogeneidade, h4: ?)
\item
  mostrar que a metodologia neoclássica não é capaz de caracterizar
  teoricamente os bens culturais:

  \begin{itemize}
  \tightlist
  \item
    função de produção, incluindo homogeneidade (p.27), utilidade
    ordinal x cardinal (impossibilidade somar utilidade ordinal) e star
    system
  \item
    leis de rendimentos marginais decrescentes
  \item
    função utilidade: consumo de duas músicas não é melhor do que de
    uma. Mostrar otimização restrita (renda e preços das músicas).
  \end{itemize}
\end{itemize}

\hypertarget{referuxeancias}{%
\section*{REFERÊNCIAS}\label{referuxeancias}}
\addcontentsline{toc}{section}{REFERÊNCIAS}

\hypertarget{refs}{}
\begin{CSLReferences}{0}{1}
\leavevmode\vadjust pre{\hypertarget{ref-bourdieu}{}}%
BOURDIEU, P. \textbf{A Economia das Trocas Simbólicas}. 6. ed. São
Paulo: Perspectiva, 2007.

\leavevmode\vadjust pre{\hypertarget{ref-herscovici}{}}%
HERSCOVICI, A. P. C. H. \textbf{Economia da Cultura e da Comunicação}.
1. ed. Vitória: Fundação Ceciliano Abel de Almeida, 1995.

\end{CSLReferences}

\end{document}
